\documentclass[letterpaper,10pt,titlepage]{article}

\usepackage{graphicx}                                        
\usepackage{amssymb}                                         
\usepackage{amsmath}                                         
\usepackage{amsthm} 
\usepackage{verbatim}                                         

\usepackage{alltt}                                           
\usepackage{float}
\usepackage{color}
\usepackage{url}

\usepackage{balance}
\usepackage[TABBOTCAP, tight]{subfigure}
\usepackage{enumitem}
\usepackage{pstricks, pst-node}
\usepackage{texments}

\usepackage{geometry}
\geometry{textheight=9in, textwidth=6.5in}

\newcommand{\cred}[1]{{\color{red}#1}}
\newcommand{\cblue}[1]{{\color{blue}#1}}

\def\braces#1{[#1]}

\usepackage{hyperref}
\usepackage{geometry}

\def\name{Christopher Martin, Jennifer Wolfe, Geoffrey Corey}

%% The following metadata will show up in the PDF properties
\hypersetup{
  colorlinks = true,
  urlcolor = black,
  pdfauthor = {\name},
  pdfkeywords = {cs472 ``computer architecture'' clements ``chapter 3''},
  pdftitle = {CS 472: Homework 3},
  pdfsubject = {CS 472: Homework 3},
  pdfpagemode = UseNone
}

\begin{document}
\usestyle{default}
\hfill \name

\hfill \today

\hfill CS 472 HW 3

\section*{Chapter 3 Homework Questions}


\begin{enumerate}
\item[$(3.1)$] \textbf{Why is the program counter a \textit{pointer} and not a \textit{counter}?}

  The program counter is used to access memory. The value held in the PC is treated as a memory address rather than an integer value.   

\item[$(3.2)$] \textbf{Explain the function of the following registers in a CPU: PC, MAR, MBR, IR}

A. PC Program Counter points to the next instruction to be executed\\
B. MAR Memory Address Register contains the memory address currently be accessed. \\
C. MBR Memory Buffer Register hold data to be written to memory or read from it.\\
D. IR Instruction Register holds the fetched, currently executing instruction
  
\item[$(3.3)$] \textbf{For each of the following 6-bit operations, calculate the values of C, Z, V, N}

\begin{tabular}{| l | l | l | l | l |}
	\hline
	X & C & Z & V & N \\ \hline
	A & 0 & 0 & 0 & 0 \\ \hline
	B & 1 & 1 & 0 & 0 \\ \hline
	C & 0 & 0 & 0 & 0 \\ \hline
	D & 1 & 0 & 0 & 0 \\ \hline 
	E & 0 & 0 & 0 & 1 \\ \hline 
	F & 1 & 0 & 0 & 1 \\ \hline 
	\end{tabular}

  
\item[$(3.10)$] \textbf{Why does ARM provide a reverse subtract instruction?}

To negate a a value because ARM does not have an actual negation instruction.

\item[$(3.17)$] \textbf{ARM uses 12-bit literal. Compare and contrast the 8-bit format and 4-bit alignment vs straight 12-bit literal.}


  
\item[$(3.18)$] \textbf{Write one or more ARM instructions that will clear bits 20 to 25 inclusive in register r0. All other bits of r0 should remain unchanged.}

\includecode{318.asm}

\item[$(3.19)$] \textbf{Swap contents of r0 and r1 without using any other registers or memory storage.}

\includecode{319.asm}

%\item[$(3.25)$] \textbf{What is teh binary encoding of the following instructions? A. STRB r1, [r2] B. LDR r3, [r4,r5]\! C. LDR r3,[r4],r5 D. LDR r3, [r4,#-6]!}
\item[$(3.25)$] \textbf{What is the binary encoding of the following instructions? A. STRB r1, \braces{r2} B. LDR r3, \braces{r4,r5}\! C. LDR r3,\braces{r4},r5 D. LDR r3, \braces{r4,\#-6}\!}

\item[$(3.39)$] \textbf{Write ARM assembly that scans a null terminal string and copies the string from a source pointed to by r0 to a destination pointed to by r1}



\item[$(3.51)$] \textbf{Write ARM assembly that determines whether an odd length string is a palindrome or not. String is ASCII encoded, stored in memory. Pointer to beginning of string in r1, pointer to end of string in r2. On exit, r0 contains 0 if not palindrome, 1 if palindrome.}



\end{enumerate}
\end{document}
