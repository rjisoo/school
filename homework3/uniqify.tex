\documentclass[letterpaper,10pt,titlepage]{article}

\usepackage{graphicx}                                        

\usepackage{amssymb}                                         
\usepackage{amsmath}                                         
\usepackage{amsthm}                                          

\usepackage{alltt}                                           
\usepackage{float}
\usepackage{color}

\usepackage{url}

\usepackage{balance}
\usepackage[TABBOTCAP, tight]{subfigure}
\usepackage{enumitem}

\usepackage{pstricks, pst-node}

\usepackage{verbatim}

\usepackage{geometry}
\geometry{textheight=10in, textwidth=7.5in}

%random comment

\newcommand{\cred}[1]{{\color{red}#1}}
\newcommand{\cblue}[1]{{\color{blue}#1}}

\usepackage{hyperref}

\def\name{Geoffrey Corey}

%pull in the necessary preamble matter for pygments output
\input{pygments.tex}

%% The following metadata will show up in the PDF properties
\hypersetup{
  colorlinks = true,
  urlcolor = black,
  pdfauthor = {\name},
  pdfkeywords = {cs311 ``operating systems'' files filesystem I/O},
  pdftitle = {CS 311 Project 2: UNIX File I/O},
  pdfsubject = {CS 311 Project 2},
  pdfpagemode = UseNone
}

\parindent = 0.0 in
\parskip = 0.2 in

\begin{document}
\section{Design}
\label{System Design & Deviations}
%\begin{Verbatim}

\subsection{The Design}
\label{DesingProcess}
The overall design for my system was very loose in the beginning of the project, mainly due to not knowing much about the
underlying system calls required for the project.

The basic design was:
	Parse input from the user
	Validate that the information supplied was valid
	begin the execute the code based upon the flag supplied
As time went by, the design was redefined. The breakdown strucutre came to:
	Tackle the -t and -v flags, after tackling parsing, opening and reading of a file
	tackle the -x and -d flags
	tackling the -A flag
\subsection{The Deviations}
\label{Deviations}
The design from the beginning was not thoughtout so thoroughly and logically that there could be much deviation from a plan. The basic structure was:
	Open file, handle errors
	read file, check correct archive
	do the required task
The only real deviations had to do with re-writing functions and adding helper functions to the main functions to process the task better.
\section{History}
\label{myar Revision History}
%\input{log.txt}

\section{Challenges}
\label{Overcoming Project challenges}
The two major obstacles in this programming project were:
	Time
	NULL terminator and strings
The biggest headache was trying to fix character bufferes to have a NULL terminator so that they could easily be printed. Couple this with diminsihng time due to other issues popping up during planned time to work on this project, the whole thing was very stressful.

\section{Questions}
\label{Project Quesions}
\subsection{The Point of the Assignment}
\label{Point}
The point of this assignment was to A) learn how to write an operating system tool, B) Learn and understand the underlying systemcalls, and C) Begin to learn how to really program in C.
\subsection{Quality Assurance}
\label{QA}
For testing myar, there was both an archive created by the Unix ar utility, and an archive created with the myar utility. Testing of the print and append were both correct and the interoperability of the two were assured. Howeve, due to time issues, the extract aand delete options were not implmented.
\subsection{What Did You Learn?}
\label{Learned}
I learned I need to allocate my time even better for future assignments if I plan to complete a fully working assignment on time.
%\end{Verbatim}


%input the pygmentized output of mt19937ar.c, using a (hopefully) unique name
%this file only exists at compile time. Feel free to change that.
\section{Code}
\label{myar Source Code}
\input{__uniqify.cpp.tex}
\end{document}
