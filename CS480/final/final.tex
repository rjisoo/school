\documentclass[letterpaper,10pt,titlepage]{article}

\usepackage{graphicx}
\usepackage{amssymb}
\usepackage{amsmath}
\usepackage{amsthm}
\usepackage{verbatim}

\usepackage{alltt}
\usepackage{float}
\usepackage{color}
\usepackage{url}

\usepackage{balance}
\usepackage[TABBOTCAP, tight]{subfigure}
\usepackage{enumitem}
\usepackage{pstricks, pst-node}

\usepackage{geometry}
\geometry{textheight=9in, textwidth=6.5in}

\newcommand{\cred}[1]{{\color{red}#1}}
\newcommand{\cblue}[1]{{\color{blue}#1}}

\def\braces#1{[#1]}

\usepackage{hyperref}
\usepackage{geometry}

\def\name{Geoffrey Corey}

%% The following metadata will show up in the PDF properties
\hypersetup{
  colorlinks = true,
  urlcolor = black,
  pdfauthor = {\name},
  pdfkeywords = {cs480 ``translators'' ``compiler optimization''},
  pdftitle = {CS 480: Final},
  pdfsubject = {CS 480: Final},
  pdfpagemode = UseNone
}

\begin{document}
%\usestyle{default}
\hfill \name

\hfill \today

\hfill CS 480 Final

\section*{Loop Optimization}
\subsection*{Overview}
In the article "A compiler optimization to reduce execution time of loop nest" 
the authors Oh-Young Kwon, Gi-Ho Park, and Tack-Don Han discuss altering
loop tiling to not tile the innermost loop, reducing the number of 
instructions for loop control and increasing performance of matrix 
multiplication. the focus of this alteration is to maximize spatial reuse of 
the innermost loop to stay within the CPU's cacheline.\\

The article first goes over a formal definition of how the algorithm works and
then explains what each of the type of vectors means in terms of loop nesting
and spacial reuse. In the testing section of the article, the authors list
what steps they took to run the code and how they went about comparing the
results of the tests. In the experiments, the authors compared their loop
tiling permutation that would take advantage of cache reuse and compared it to
the conventional loop tiling algorithm for matrix multiplication.\\

The results of the tests were shown with 2 different CPU architectures:
DEC 5000/25 and a SPARC 10. When only using the permutation algorithm, there
was 1.44 and 2.61 fold computation time improvement on each architecture
respectively. However, when applying the tiling technique, the execution time
improvement is only 1.92 and 1.08 fold respectively.

\subsection*{Critique}
This article is aimed at people who know more about loop nesting and know more
about loop nesting and using loop tiling to decrease execution time of nested
loops. The authors do a decent job of trying to explain what loop tiling is
though and they do a even better job of explaining the shortcomings of the
standard loop tiling algorithm.\\

In the algorithm explanation, the authors introduces the term "cacheline", but
does not seem to define to the reader what a CPU chaceline is. The only thing
the reader can concretely draw from this term is that it has something to do
with the CPU's cache, but without the definition, it can obfuscate what the
advantages of what spatial and temporal reuse in the algorithm can actually
accomplish in speeding up the execution of the matrix multiplication example.
It is not until later in Section 4 that the authors describe what a 
"cacheline" is, but even then they still do not relate it back to the term.\\

For the proof of the effectiveness of the algorithm, the authors does not show
the average time of execution for X number of tests. They also do not detail
how they generated the numbers to fill the matrices for the matrix 
multiplication. However, the authors did keep timings of the execution times
as they increased the size of the arrays to track whether the observed 
execution time decrease had a plateau.

Overall, the authors do clearly show that their nested loop modification does
increase the performance of matrix multiplication and they do relate the
results of the timings back to the reason of taking advantage of CPU cache 
access time.

\section*{Function Inlining and Loop Unrolling}
\subsection*{Overview}
\subsection*{Critique}
Does a good job explaining what inlining and loop unrolling is, provides pros
and cons analysis.

\bibliographystyle{unsrt}
\bibliography{cites}
\end{document}
