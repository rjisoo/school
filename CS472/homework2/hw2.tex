\documentclass[letterpaper,10pt,titlepage]{article}

\usepackage{graphicx}                                        
\usepackage{amssymb}                                         
\usepackage{amsmath}                                         
\usepackage{amsthm} 
\usepackage{verbatim}                                         

\usepackage{alltt}                                           
\usepackage{float}
\usepackage{color}
\usepackage{url}

\usepackage{balance}
\usepackage[TABBOTCAP, tight]{subfigure}
\usepackage{enumitem}
\usepackage{pstricks, pst-node}

\usepackage{geometry}
\geometry{textheight=9in, textwidth=6.5in}

%random comment

\newcommand{\cred}[1]{{\color{red}#1}}
\newcommand{\cblue}[1]{{\color{blue}#1}}

\usepackage{hyperref}
\usepackage{geometry}

\def\name{Christopher Martin, Jennifer Wolfe, Geoffrey Corey}

%% The following metadata will show up in the PDF properties
\hypersetup{
  colorlinks = true,
  urlcolor = black,
  pdfauthor = {\name},
  pdfkeywords = {cs472 ``computer architecture'' clements ``chapter 2''},
  pdftitle = {CS 472: Homework 2},
  pdfsubject = {CS 472: Homework 2},
  pdfpagemode = UseNone
}

\begin{document}
\hfill \name

\hfill \today

\hfill CS 472 HW 2

\begin{enumerate}
\item[$(2.5)$] \textbf{Calculations are to be performed to a percision of $0.001$\%. How many bits does this require?
}

  %Answer goes here -- make sure you put a blank line between question and answer.   

\item[$(2.13)$] \textbf{Perofrm the following calculations in the stated bases.}
\subitem{$(a)$ $00110111_{2} + 01011011_{2}$}
\subsubitem{$10001000_{2}$}
\subitem{$(b)$ $00111111_{2} + 01001001_{2}$}
\subsubitem{$10001000_{2}$}
\subitem{$(c)$ $00120121_{16} + 0A015031_{16}$}
\subsubitem{$0A135152_{16}$}
\subitem{$(d)$ $00ABCD1F_{16} + 0F00800F_{16}$}
\subsubitem{$0FAC4D2E_{16}$}

  
\item[$(2.14)$] \textbf{What is artihmetic overflow? When does it occur and how can it be detected?}

  When the sign bit is the opposite of what is expected. Accounting: reverse sign or change data formats.
  
\item[$(2.16)$] \textbf{Convert $1234.125$ to $32$-bit IEEE floating-point format.}

  $01000100100110100100100000000000$  
  
\item[$(2.17)$] \textbf{What is the decimal equivalent of the $32$-bit IEEE floating-point value $CC4C0000$?}

  $-53477376$

\item[$(2.22)$] \textbf{What is the diefference between a \textit{truncation} error and a \textit{rounding} error?}

  Truncation error will always be lower than the expected value and less percise since the trailing bits are ignored.
  Rounding error will tend to be mor percise since the trailing bits are considered, and the resulting value wiill be lower or greater based upon the chosen algorithm.
  
\item[$(2.40)$] \textbf{Draw the truth table for the figure P$2.40$ and explain what it does.}

  This logic implents the same outputs as an XOR gate.
\begin{center}
	\begin{tabular}{ | l  l || l  l  l || l |}
	\hline
	A & B & P & Q & R & C \\ \hline
	0 & 0 & 1 & 1 & 1 & 0 \\ \hline
	0 & 1 & 1 & 1 & 0 & 1 \\ \hline
	1 & 0 & 1 & 0 & 1 & 1 \\ \hline
	1 & 1 & 0 & 1 & 1 & 1 \\ \hline 
	\end{tabular}
\end{center}



\item[$(2.45)$] \textbf{Is it possible to have \textit{n}-input  AND, OR, and NOR gates where \textit{n}$>2$? Explain your answer with a truth table.}

  Yes it is.

  EX: A and B go to a XOR, the output (X) and C go to an XOR. Output is Y.
\begin{center}
	\begin{tabular}{ | l l l || l  l |}
	\hline
	A & B & C & X & Y \\ \hline
	0 & 0 & 0 & 0 & 0 \\ \hline
	0 & 0 & 1 & 0 & 1 \\ \hline
	0 & 1 & 0 & 1 & 1 \\ \hline
	0 & 1 & 1 & 1 & 0 \\ \hline 
	1 & 0 & 0 & 1 & 1 \\ \hline
	1 & 0 & 1 & 1 & 0 \\ \hline
	1 & 1 & 0 & 0 & 0 \\ \hline
	1 & 1 & 1 & 0 & 1 \\ \hline 
	\end{tabular}
\end{center}

\end{enumerate}



\end{document}
