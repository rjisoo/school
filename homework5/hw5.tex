\documentclass[letterpaper,10pt,titlepage]{article}

\usepackage{graphicx}                                        
\usepackage{amssymb}                                         
\usepackage{amsmath}                                         
\usepackage{amsthm} 
\usepackage{verbatim}                                         

\usepackage{alltt}                                           
\usepackage{float}
\usepackage{color}
\usepackage{url}

\usepackage{balance}
\usepackage[TABBOTCAP, tight]{subfigure}
\usepackage{enumitem}
\usepackage{pstricks, pst-node}
\usepackage{texments}

\usepackage{geometry}
\geometry{textheight=9in, textwidth=6.5in}

\newcommand{\cred}[1]{{\color{red}#1}}
\newcommand{\cblue}[1]{{\color{blue}#1}}

\def\braces#1{[#1]}

\usepackage{hyperref}
\usepackage{geometry}

\def\name{Christopher Martin, Jennifer Wolfe, Geoffrey Corey}

%% The following metadata will show up in the PDF properties
\hypersetup{
  colorlinks = true,
  urlcolor = black,
  pdfauthor = {\name},
  pdfkeywords = {cs472 ``computer architecture'' clements ``chapter 9''},
  pdftitle = {CS 472: Homework 5},
  pdfsubject = {CS 472: Homework 5},
  pdfpagemode = UseNone
}

\begin{document}
\usestyle{default}
\hfill \name

\hfill \today

\hfill CS 472 HW 5

\section*{Chapter 9	 Homework Questions}


\begin{enumerate}
\item[$(9.2)$] \textbf{Why do computers use cache memory?}

To increase performance.

\item[$(9.3)$] \textbf{}

\textbf{Temporal Locality} - addresses that are accessed over and over again within a short time span.\\
\textbf{Spacial Locality} - addresses that are clustered within the same region of memory.

\item[$(9.4)$] \textbf{}

\item[$(9.5)$] \textbf{}

$t_m$ = 70ns, $t_c$ = 7ns, h = 0.9
S = 5.3\\

$t_m $ = 60ns, $t_c$ = 3ns, h = 0.8
S = 6.9\\

$t_m$ = 60ns, $t_c$ = 3ns, h = 0.8
S= 4.2\\

$t_m$ = 60ns, $t_c$ = 3ns, h = 0.97
S = 7.2

\item[$(9.6)$] \textbf{}

$t_m$ = 60ns, $t_c$ = 3ns, S = 1.1
h = 0.09\\

$t_m$ = 60ns, $t_c$ = 3ns, S = 2.0
h = 0.52\\

$t_m$ = 60ns, $t_c$ = 3ns, S = 5.0
h = 0.84\\

$t_m$ = 60ns, $t_c$ = 3ns, S = 15.0
h = 0.98

\item[$(9.8)$] \textbf{Calculate the maximum speed-up ratio you could expect to see as \textit{h} approaches 100\%.}

\item[$(9.11)$] \textbf{In a direct mapped cache memory system, what is the meaning of the following terms: Word, Line, Set.}

Word: 16-bit or 32-bit, Line: made up of individual words, Set: Units of lines.

\item[$(9.12)$] \textbf{}

\item[$(9.17)$] \textbf{}

Implies that data in the various cache and memories is not stale and up-to-date.
	
\item[$(9.22)$] \textbf{}

Because contents of memory cache are not modified.

\item[$(9.23)$] \textbf{}

\item[$(9.26)$] \textbf{}

\item[$(9.28)$] \textbf{}

\item[$(9.35)$] \textbf{}

\item[$(9.41)$] \textbf{}

\item[$(9.42)$] \textbf{}

\item[$(9.43)$] \textbf{}

\item[$(9.45)$] \textbf{A compute runs an instruction set with the with characteristics in a table. What is the average number of cycles per instruction?}

\item[$(9.46)$] \textbf{}

\item[$(9.57)$] \textbf{}

\end{enumerate}
\end{document}
